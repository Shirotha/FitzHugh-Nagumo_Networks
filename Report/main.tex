\documentclass[aps,pra,%twocolumn,
					%preprint,
					showpacs,superscriptaddress,nofootinbib]{revtex4-1}


%\bibliographystyle{apsrev}
\usepackage{helvet}
\usepackage{eurosym}
\usepackage{color}
\usepackage[utf8]{inputenc}
%\usepackage{subfigure}
\usepackage{amssymb}
\usepackage{amsmath}
\usepackage{graphicx}
\usepackage{bm}
\usepackage{xcolor}

\usepackage{natbib}
\usepackage{url}
\usepackage{textcase}
\usepackage{bm}

\usepackage{placeins}

\usepackage{amsfonts}
\usepackage{mathtools}
\usepackage{calrsfs}
\usepackage{braket}
\usepackage[hidelinks]{hyperref}
\usepackage{subfig}

\DeclareMathAlphabet{\pazocal}{OMS}{zplm}{m}{n}
\DeclareGraphicsExtensions{%
 .png,%
 .jpg,%
 .pdf,.PDF,%
 .mps,.jpeg,.jbig2,.jb2,.JPG,.JPEG,.JBIG2,.JB2}

%some definitions

\makeatletter


\newcommand{\abs}[1]{\left| #1 \right|} % for absolute value
\newcommand{\avg}[1]{\left< #1 \right>} % for average
\def\d{\mathrm d}
\renewcommand{\dfrac}[2]{\frac{\d #1}{\d #2}} % for derivatives
\newcommand{\ddfrac}[2]{\frac{\d^2 #1}{\d #2^2}} % for double derivatives
\newcommand{\pdfrac}[2]{\frac{\partial #1}{\partial #2}}
\newcommand{\pddfrac}[2]{\frac{\partial^2 #1}{\partial {#2}^2}}
\newcommand{\pdddfrac}[2]{\frac{\partial^2 #1}{\partial {#2}^2}}
%\newcommand{\ket}[1]{\left| #1 \right>} % for Dirac bras
%\newcommand{\bra}[1]{\left< #1 \right|} % for Dirac kets
%\newcommand{\braket}[2]{\left< #1 \vphantom{#2} \right|
% \left. #2 \vphantom{#1} \right>} % for Dirac brackets
%\newcommand{\ketbra}[2]{\left| #1 \vphantom{#2} \right>
% \left< #2 \vphantom{#1} \right|} % for Dirac brackets
\newcommand{\braces}[1]{\left( #1 \right)}
\newcommand{\brackets}[1]{\left[ #1 \right]}
\newcommand{\curlys}[1]{\left\{ #1 \right\}}
\newcommand{\Braces}[1]{\Big( #1 \Big)}
\newcommand{\Brackets}[1]{\Big[ #1 \Big]}
\newcommand{\Curlys}[1]{\Big\{ #1 \Big\}}
\newcommand{\commut}[1]{\left[ #1 \right]}
\newcommand{\super}[1]{{\pazocal{#1}}}
\newcommand{\Tr}[1]{\text{Tr}}
\newcommand{\Exp}[1]{\text{e}^{#1}}

\newcommand{\add}[1]{\textcolor{blue}{#1}} % add new text
\newcommand{\rep}[2]{\textcolor{red}{#1}\ \textcolor{blue}{#2}} % replace with new text
\newcommand{\rem}[1]{\textcolor{red}{#1}} % remove text
\newcommand{\Avg}[1]{\langle{#1}\rangle} % for average

\def\ii{i}
\def\to{\rightarrow}

% args: width, file
\newcommand{\Image}[2]{\includegraphics[width=#1\textwidth]{#2}}

% args: label, width, file, caption
\newcommand{\Figure}[4]{\begin{figure}[!ht]
	\centering
	\Image{#2}{#3}
	\caption{#4}
	\label{fig:#1}
\end{figure}}

% args: label, width1, file1, caption1, width2, file2, caption2, caption
\newcommand{\FigureRow}[8]{\begin{figure}[!ht]
	\centering
	\subfloat[#4\label{sfig:#1-1}]{\Image{#2}{#3}}
	\hfill
	\subfloat[#7\label{sfig:#1-2}]{\Image{#5}{#6}}
	\caption{#8}
	\label{fig:#1}
\end{figure}}

\newcommand{\FigureGrid@Cont}[7]{\begin{figure}[!ht]
	\centering
	\subfloat[\arg@captionA\label{sfig:\arg@label-1}]{\Image{\arg@widthA}{\arg@fileA}}
	\hfill
	\subfloat[\arg@captionB\label{sfig:\arg@label-2}]{\Image{\arg@widthB}{\arg@fileB}} \\
	\subfloat[#3\label{sfig:\arg@label-3}]{\Image{#1}{#2}}
	\hfill
	\subfloat[#6\label{sfig:\arg@label-4}]{\Image{#4}{#5}} 
	\caption{#7}
	\label{fig:\arg@label}
\end{figure}}

\newcommand{\FigureGrid}[7]{%
	\def\arg@label{#1}
	\def\arg@widthA{#2}
	\def\arg@fileA{#3}
	\def\arg@captionA{#4}
	\def\arg@widthB{#5}
	\def\arg@fileB{#6}
	\def\arg@captionB{#7}
	\FigureGrid@Cont%
}

% args: label
\def\fig<#1>{Fig. \ref{fig:#1}}

% args: label, number
\def\sfig<#1,#2>{Fig. \ref{sfig:#1-#2}}

% args: label, equation
\newcommand{\Equation}[2]{\begin{equation}
	\label{eq:#1}
	\begin{aligned}
		#2
	\end{aligned}
\end{equation}}

% args: label
\def\eq<#1>{Eq. (\ref{eq:#1})}

\def\SectionBreak{\FloatBarrier\newpage}

\makeatother

\begin{document}

	\thispagestyle{empty}

\begin{center}

\rule{0mm}{3cm}

\textbf{\LARGE Coherence resonance in heterogeneous networks}

\vspace{4mm}

Chiara Mielau, Alexander Solovev, Vincent Stegmaier, Sven Zimmermann

\vspace{2mm}

Betreuer: Leonhard Schülen

\vspace{4cm}

\begin{minipage}[c]{0.7\textwidth}
	\begin{center}
		\textbf{\large Task}
	\end{center}
	The counter-intuitive effect of coherence resonance describes a non-monotonic behavior of the
	regularity of noise-induced oscillations in the excitable regime, leading to an optimum response
	in terms of regularity of the excited oscillations for an intermediate noise strength \cite{Pikovsky1997}. Coherence
	resonance has been investigated in complex networks with local, nonlocal and global topologies as
	well as in random and small-world networks \cite{Masoliver2017}. The phenomenon of coherence resonance has been
	also recently studied in a two-layer multiplex network \cite{Semenova2018}. In \cite{Masoliver2017} and \cite{Semenova2018} the network elements are
	all assumed to be identical. In real-world systems, however, the elements are often non-identical.
	The goal of this project is to investigate the impact of heterogeneity on coherence resonance.
	
	\begin{itemize}
		\item Perform a literature search about this topic. Has coherence resonance been previously investigated in networks with heterogeneous elements?
		\item Reproduce the results (without delays) obtained in \cite{Masoliver2017}. In particular, reproduce Fig. 2.
		\item Consider the case of inhomogeneous FitzHugh-Nagumo neurons in excitable regime. For example, $a_i$ is normally distributed around the value 1.1. 
			Make sure that that $a_i > 1 \forall i$ as this is the bifurcation point. Perform a simulation for $\sigma = 0.1$ and $D = 0.001$. What are the effects of the inhomogeneities?
		\item  Try to get a similar curve as in Fig. 2, now for the inhomogeneities. Is the picture similar or does it deviate significantly?
			 If so, is the effect dependent on the noise intensity/the coupling strength?
	\end{itemize}
\end{minipage}

\end{center}

\newpage

\tableofcontents
\thispagestyle{empty}

\setcounter{page}{0}
	\SectionBreak
	\section{Theory}
\subsection{FitzHugh-Nagumo model}

Richard FitzHugh \cite{FitzHugh1961} and J. Nagumo \cite{Nagumo1962} developed almost simultaneously a model to describe the excitation of a neuron mathematically. Starting from van der Pol oscillator

\Equation{vdp}{
	\frac{d^2x}{dt^2} - \mu (1-x^2) \frac{dx}{dt} + x = 0
}

%\begin{align}
%    \frac{d^2x}{dt^2} - \mu (1-x^2) \frac{dx}{dt} + x = 0
%\label{eqn:vdp}
%\end{align}

R. FitzHugh derived a more general form. To do so, he carried out a Lienard transform\cite{Lienard1928} of \eq<vdp> and added some constants to obtain two coupled differential equations describing an activator variable $u(t)$ and a recovery-like variable $v(t)$: 

\Equation{fhnu}{
	\epsilon \frac{du}{dt} = u - \frac{u^3}{3} - v
}

%\begin{align}
%    \epsilon \frac{du}{dt} = u - \frac{u^3}{3} - v
%\label{eqn:fhnu}
%\end{align}

\Equation{fhnv}{
	\frac{dv}{dt} = u + a + \sqrt{2D} \xi (t)
}

%\begin{align}
%    \frac{dv}{dt} = u + a + \sqrt{2D} \xi (t)
%\label{eqn:fhnv}
%\end{align}

The excitation behavior in \eq<fhnu> evolves on a faster time scale than the inhibitor variable $v$ which is reached by $\epsilon < 1$. The constant $a$ denotes the bifurcation parameter for which $a > 1 $ describes the excitable regime. Since we are studying systems with background noise the last term in \eq<fhnv> is added to the recovery variable where $D$ denotes the noise intensity. $\xi (t)$ is chosen as Gaussian white noise with $\langle \xi (t) \rangle = 0$ and $\langle \xi (t + \tau) \xi (t) \rangle = \delta (\tau)$. 
\newline
\\
The FitzHugh-Nagumo (FHN) model can now be applied to several coupled oscillators. \eq<fhnu> and \eq<fhnv> will be slightly modified by an additional coupling term in the activation differential equation (see \eq<fhnu>), where $N$ denotes the number of oscillators in system,

\Equation{fhnui}{
	\epsilon \frac{du_i}{dt} = u_i - \frac{u_i^3}{3} - v_i + \frac{\sigma}{2P} \sum_{j=i-P}^{i+P} (u_i(t) - u_{i-1}(t)) \ , \ i=1...N
}

%\begin{align}
%    \epsilon \frac{du_i}{dt} = u_i - \frac{u_i^3}{3} - v_i + \frac{\sigma}{2P} \sum_{j=i-P}^{i+P} (u_i(t) - u_{i-1}(t)) \ , \ i=1...N
%\end{align}

to obtain an equation for a ring coupled network of FHN oscillators. The topology of the system is defined by the constant $P$ that declares the number of next neighbours coupled to each oscillator. The global coupling strength is denoted by $\sigma$.   

\subsection{Coherence Resonance}

Coherence Resonance describes the regularity of excitations over time in noise driven systems such as FitzHugh-Nagumo model \cite{Pikovsky1997}. The more regular the spikes occur the more equidistant is the inter-spike-intervall $t_{ISI}$ (see \fig<topology-CR>b). There are several different ways of describing how coherent a system is. In this project we focus on the normalized standard deviation of the inter-spike-interval $t_{ISI}$, so called $R_T$,

\Equation{rt0}{
	R_T = \frac{\sqrt{\langle t_{ISI}^2 \rangle - \langle t_{ISI} \rangle ^2}}{\langle t_{ISI} \rangle}	\nonumber
}

%\begin{align*}
%    R_T = \frac{\sqrt{\langle t_{ISI}^2 \rangle - \langle t_{ISI} \rangle ^2}}{\langle t_{ISI} \rangle}
%\end{align*}

where $\langle ... \rangle$ denote the average over time $t$. This equation holds for single FitzHugh-Nagumo oscillators. In case of a network of $N \geq 2$ FHN oscillators the average over each node has to be taken into account 

\Equation{rt}{
	R = \frac{\sqrt{\langle \overline{t_{ISI}^2} \rangle - \langle \overline{t_{ISI}}  \rangle ^2}}{\langle \overline{t_{ISI}}  \rangle}
}

%\begin{align}
%    R = \frac{\sqrt{\langle \overline{t_{ISI}^2} \rangle - \langle \overline{t_{ISI}}  \rangle ^2}}{\langle \overline{t_{ISI}}  \rangle}
%\end{align}


\section{Coherence Resonance in homogeneous networks}

To encounter the effect of heterogeneity on the network we first focus on the homogeneous case  for a later reference. The simulations were carried out for next-neighbour coupling $P=1$ (see \sfig<topology-CR,1>) with fixed coupling strength. \fig<space-time> shows the activation variable $u(t)$ over time for each node in the network. For evaluating the optimal noise and coupling strength for coherence resonance we simulated the system once for fixed coupling strength and variable noise intensity and once for fixed noise intensity and variable coupling strength. Furthermore, both simulation types were done for two different bifurcation parameters (see \fig<CR-plots>). This reflects the non-monotonic behaviour as shown in \cite{Masoliver2017}. 

\FigureRow{topology-CR}
	{0.5}{image/ring_network_20}{Ring-Network topology for 20 next neighboured coupled FitzHugh-Nagumo oscillators.}
	{0.5}{image/definition_CR}{From \cite{Pikovsky1997}: Spiking neurons over time. $t_{ISI}$ is defined here as $t_P$.}{
	Visualization of the network topology (a) and time evolution of a single oscillator (b).
}

\newpage

\Figure{space-time}{0.8}{image/space_time_homo}{
	From \cite{Masoliver2017}: Space time diagram for different noise intensities (a) $D = 0.00012$, (b) $D = 0.001$, (c) $D = 0.005$, (d) $D = 0.05$ for $a = 1.05$, $\sigma = 0.1$, $\epsilon = 0.01$
}

As one can see in \fig<CR-plots> the minimum meaning the most coherent state is shifted to higher values of noise and coupling strength respectively. In addition to that, the shape of the function near the minimum is sharpened. 

\Figure{CR-plots}{0.8}{image/CR_homo}{
	From \cite{Masoliver2017}: Measure of coherence Resonance for different noise intensities (left) and coupling strength (right), both plotted logarithmic. The yellow triangles denote the results for $a = 1.3$ and blue circles denote $a = 1.05$
}
	%\SectionBreak
	%\input{Reproduction?}
	\SectionBreak
	\section{Heterogeneous bifurcation parameter}
\label{sec:hetero-a}

To implement the effect of heterogeneity we introduce a position dependent bifurcation parameter as shown in \eq<ai>:

\Equation{ai}{
	a_i = a + \delta a \chi_i
}

where $a$ is the mean bifurcation parameter of the system, $\delta a$ is the heterogeneity and $\chi_i$ ist Gaussian white noise with zero mean.

In the case of a heterogeneous system $\delta a > 0$ the grade of coherence changes in dependence of said heterogeneity $\delta a$
in different ways for different system parameters like diffusion constant $D$ as shown in \fig<hetero-a-noise> and
coupling strength $\sigma$ as shown in \fig<hetero-a-coupling>.

These figures were obtained by simulating the oscillator network over a constant time period resulting in varying amounts of measured excitations
and therefor a varying degree of accuracy for the normalized standard deviation of the inter-spike-interval $R_T$ especially for low noise
and stronger coupling resulting in a higher spread of datapoints in these regimes.

As seen in \sfig<hetero-a-noise,1> the heterogeneity has a positive effect on the coherence, lowering the value of $R_T$ for low noise values
and effectively increasing the noise which also increases the frequency of excitation
and in turn shifts the optimal noise towards lower noises, though the effect vanishes for larger noises where the coherence becomes independent on the heterogeneity.

This effect is only visible close to the bifurcation since going farther away from it -- towards larger values of $a$ -- reverses it as seen in \sfig<hetero-a-noise,2>,
shifting the optimal noise towards higher noises instead, additionally the best reachable coherence also gets worse.
Nevertheless the dependence of the coherence from the heterogeneity is still only present for low noises.

\FigureRow{hetero-a-noise}{0.5}{image/plotDR_a105}{$a = 1.05$}{0.5}{image/plotDR_a13}{$a = 1.3$}{
	The Normalized standard deviation of the inter-spike-interval $R_T$ plotted over different diffusion coefficients $D$
	for small and large bifurcation parameter $a$ and for different heterogeneity $\delta a$,
	For constant coupling strength $\sigma = 0.1$ and timescale parameter $\epsilon = 0.01$.
	Simulation for $N = 100$ oscillators and constant simulation time $t = 1000$.
}

Looking at the changes over different coupling strengths $\sigma$ yields similar results. As seen in \sfig<hetero-a-coupling,1> with increasing heterogeneity $\delta a$ the
plateau of optimal coherence gets smaller as the center is shifted towards lower coupling strength but maintaining the position of its left border
and raising the plateau at low coupling strengths. Farther away from the bifurcation the effect is again reversed -- as seen in \sfig<hetero-a-coupling,2> --
as the minimum gets shifted towards higher coupling strength and the general value of coherence lessens.

\FigureRow{hetero-a-coupling}{0.5}{image/plotsR_a105}{$a = 1.05$}{0.5}{image/plotsR_a13}{$a = 1.3$}{
	The Normalized standard deviation of the inter-spike-interval $R_T$ plotted over different coupling strengths $\sigma$
	for small and large bifurcation parameter $a$ and for different heterogeneity $\delta a$,
	for optimal diffusion constant $D_{opt}$ and constant timescale parameter $\epsilon = 0.01$.
	Simulation for $N = 100$ oscillators and constant simulation time $t = 1000$.
}

In \fig<compare-space-time> an overview of these effects is given.
The top row shows that with heterogeneity (\sfig<compare-space-time,2>) there are more excitations but they start at multiple locations instead
at only one as seen in \sfig<compare-space-time,1>.
For higher bifurcation parameter -- see \sfig<compare-space-time,3> -- the amount of excitations are lower and they don't form waves anymore,
but the heterogeneity still increases the amount as shown in \sfig<compare-space-time,4>.

\FigureGrid{compare-space-time}
	{0.45}{image/plotXTD000012a105da0}{$a = 1.05, D = 0.00012, \delta a = 0$}
	{0.45}{image/plotXTD000012a105da005}{$a = 1.05, D = 0.00012, \delta a = 0.05$}
	{0.45}{image/plotXTD0004a13da0}{$a = 1.3, D = 0.004, \delta a = 0$}
	{0.45}{image/plotXTD0004a13da005}{$a = 1.3, D = 0.004, \delta a = 0.05$}{
	Space-Time plots for the homogeneous system (left) and a heterogeneous system (right) and
	low Bifurcation parameter (top) and high Bifurcation parameter (bottom) for the same parameters of
	coupling strength $\sigma = 0.1$ and timescale parameter $\epsilon = 0.01$.
}
	\SectionBreak
	

%\begin{figure}[!ht]
%	\centering
%	\subfloat[\label{sfig:DR_105}]{\includegraphics[width=0.5\textwidth]{a105RD}}
%	\hfill
%	\subfloat[\label{sfig:sR_105}]{\includegraphics[width=0.5\textwidth]{a13RD}}
%	\caption{The Normalized standard deviation of the inter-spike-interval $R_T$ plotted over the diffusion parameter \textit{D}
%for small and large bifurcation parameter a and for different heterogeneity \textit{$\delta a$}, For constant coupling strenght \textit{$\sigma=0.1$}
%and constant timescale parameter $\epsilon$ = 0.01. Simulation for \textit{N} = 100 oscillators and constant simulation time
%\textit{t} = 1000.}
%	\label{fig:RD}
%\end{figure}
%
%\begin{figure}[!ht]
%	\centering
%	\subfloat[\label{sfig:DR_13}]{\includegraphics[width=0.5\textwidth]{a105RS}}
%	\hfill
%	\subfloat[\label{sfig:sR_13}]{\includegraphics[width=0.5\textwidth]{a13RS}}
%	\caption{The Normalized standard deviation of the inter-spike-interval $R_T$ plotted over the diffusion parameter \textit{D}
%for small and large bifurcation parameter a and for different heterogeneity \textit{$\delta a$}, For constant coupling strenght \textit{$\sigma=0.1$}
%and constant timescale parameter $\epsilon$ = 0.01. Simulation for \textit{N} = 100 oscillators and constant simulation time
%\textit{t} = 1000.}
%	\label{fig:RS}
%\end{figure}








\section{Summery and Conclusion}

At first we showed the theoretical construct of the FitzHugh-Nagumo model for one oscillator. Further we also showed the theory for N ring networks. 
Afterwards  we reproduced normalized standard deviation of
the inter-spike-interval $R_T$. Therefor we looked at the effects by varying the bifurcation parameter \textit{a}, heterogeneity term \textit{$\delta a$}, the diffusion term \textit{D} and the coupling strength \textit{$\sigma$}.
By simulating the normalized standard deviation of
the inter-spike-interval $R_T$ in respect to the diffusion term \textit{D} and the coupling strength \textit{$\sigma$} for different bifurcation parameter \textit{a} we could determine what the optimal parameter are to achieve coherence resonance.\\
First we had to compare our simulation without heterogeneous terms to the Fig. 2 of the paper by Masoliver et al \cite{Masoliver2017}, which was successful. Furthermore we had the task to integrate an inhomogeneous term. This is shown in \eq<ai>. In \fig<hetero-a-noise> and \fig<hetero-a-coupling> are the plots  for the varying parameters.\\
In \sfig<hetero-a-noise,2> is shown that for higher bifurcation parameter \textit{a} the diffusion term has to increase to achieve coherence resonance. There is also the effect that with increasing \textit{a} the normalized standard deviation of
the inter-spike-interval $R_T$ increases. Since this is the degree of how coherent the network is it worsens. \\
With the inhomogeneity term for $a=1.05$ there is almost non shift. So the value of $R_T$ stays constant and the optimal noise intensity shifts a little towards smaller diffusion terms. Further away from the bifurcation point with expending \textit{$\delta a$} it shifts toward higher  \textit{D} and $R_T$.\\

\Figure{min}{0.6}{image/plotMinDRs}{
	Plotted optimal diffusion coefficient \textit{$D_{opt}$} in the $D$--$R_T$--space,
	for small and large bifurcation parameter \textit{a} and for different heterogeneity \textit{$\delta a$}, constant coupling strenght \textit{$\sigma=0.1$}
	and constant timescale parameter $\epsilon$ = 0.01. Simulation for \textit{N} = 100 oscillators and constant simulation time
	\textit{t} = 1000.
}

%\begin{figure}
%\centering
%\includegraphics[scale=0.5]{min}
%\caption{Plotted optimal diffusion coefficient \textit{$D_{opt}$} in the $D$--$R_T$--space,
%for small and large bifurcation parameter \textit{a} and for different heterogeneity \textit{$\delta a$}, constant coupling strenght \textit{$\sigma=0.1$}
%and constant timescale parameter $\epsilon$ = 0.01. Simulation for \textit{N} = 100 oscillators and constant simulation time
%\textit{t} = 1000.}
%\label{min}
%\end{figure}

We also plotted the optimal diffusion terms \textit{$D_{opt}$} for the varying bifurcation parameter (lines) with the varying inhomogeneous terms (dots) in \fig<min>. The dots represent the minima of the \fig<hetero-a-noise> and \fig<hetero-a-noise-extra> (appendix).
Here is also to denote that the color of the dots is same as for each inhomogeneous and homogeneous data set. The lines represent to which bifurcation parameter the dots belong.
So from \fig<min> is to obtain that the best bifurcation value is $a=1.05$ because the normalized standard deviation of
the inter-spike-interval $R_T$ is the lowest. When the bifurcation parameter increases the $R_T$ value rises and the inhomogeneity spreads.\\

Further we also simulated the normalized standard deviation of
the inter-spike-interval $R_T$ over the coupling strength \textit{$\sigma$} see \fig<hetero-a-coupling>. Also here we obtained similar results. With higher bifurcation parameter \textit{a} the coupling strength \textit{$\sigma$} has to increase to achieve coherence resonance. Also the degree how coherent the system is gets worse. So also here it $a=1.05$ is optimal. With integrating inhomogeneous terms there will be lesser coupling strength needed as for \textit{$\delta a=0$}. For a bifurcation term $a=1.3$ it is the other way around.\\


 


	\SectionBreak
	\section{Appendix}

%\begin{figure}[!ht]
%	\centering
%	\subfloat[\label{sfig:DR_11}]{\includegraphics[width=0.5\textwidth]{plotDR_a11}}
%	\hfill
%	\subfloat[\label{sfig:sR_11}]{\includegraphics[width=0.5\textwidth]{plotsR_a11}}
%	\caption{The Normalized standard deviation of the inter-spike-interval $R_T$ plotted over the diffusion parameter \textit{D}
%for small and large bifurcation parameter a and for different heterogeneity \textit{$\delta a$}, For constant coupling strenght \textit{$\sigma=0.1$}
%and constant timescale parameter $\epsilon$ = 0.01. Simulation for \textit{N} = 100 oscillators and constant simulation time
%\textit{t} = 1000.}
%	\label{fig:a11}
%\end{figure}
%
%\begin{figure}[!ht]
%	\centering
%	\subfloat[\label{sfig:DR_12}]{\includegraphics[width=0.5\textwidth]{plotDR_a12}}
%	\hfill
%	\subfloat[\label{sfig:sR_12}]{\includegraphics[width=0.5\textwidth]{plotsR_a12}}
%	\caption{The Normalized standard deviation of the inter-spike-interval $R_T$ plotted over the diffusion parameter \textit{D}
%for small and large bifurcation parameter a and for different heterogeneity \textit{$\delta a$}, For constant coupling strenght \textit{$\sigma=0.1$}
%and constant timescale parameter $\epsilon$ = 0.01. Simulation for \textit{N} = 100 oscillators and constant simulation time
%\textit{t} = 1000.}
%	\label{fig:a11}
%\end{figure}

\FigureRow{hetero-a-noise-extra}{0.5}{image/plotDR_a11}{$a = 1.1$}{0.5}{image/plotDR_a12}{$a = 1.2$}{
	The Normalized standard deviation of the inter-spike-interval $R_T$ plotted over different diffusion coefficients $D$
	for small and large bifurcation parameter $a$ and for different heterogeneity $\delta a$,
	For constant coupling strength $\sigma = 0.1$ and timescale parameter $\epsilon = 0.01$.
	Simulation for $N = 100$ oscillators and constant simulation time $t = 1000$.
}

\FigureRow{hetero-a-coupling-extra}{0.5}{image/plotsR_a11}{$a = 1.1$}{0.5}{image/plotsR_a12}{$a = 1.2$}{
	The Normalized standard deviation of the inter-spike-interval $R_T$ plotted over different coupling strengths $\sigma$
	for small and large bifurcation parameter $a$ and for different heterogeneity $\delta a$,
	for optimal diffusion constant $D_{opt}$ and constant timescale parameter $\epsilon = 0.01$.
	Simulation for $N = 100$ oscillators and constant simulation time $t = 1000$.
}

\bibliography{lit}
	
\end{document}
