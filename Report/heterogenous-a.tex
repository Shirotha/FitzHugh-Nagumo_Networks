\section{Heterogeneous bifurcation parameter}
\label{sec:hetero-a}

To implement the effect of heterogeneity we introduce a position dependent bifurcation parameter as shown in \eq<ai>:

\Equation{ai}{
	a_i = a + \delta a \chi_i
}

where $a$ is the mean bifurcation parameter of the system, $\delta a$ is the heterogeneity and $\chi_i$ ist Gaussian white noise with zero mean.

In the case of a heterogeneous system $\delta a > 0$ the grade of coherence changes in dependence of said heterogeneity $\delta a$
in different ways for different system parameters like diffusion constant $D$ as shown in \fig<hetero-a-noise> and
coupling strength $\sigma$ as shown in \fig<hetero-a-coupling>.

These figures were obtained by simulating the oscillator network over a constant time period resulting in varying amounts of measured excitations
and therefor a varying degree of accuracy for the normalized standard deviation of the inter-spike-interval $R_T$ especially for low noise
and stronger coupling resulting in a higher spread of datapoints in these regimes.

As seen in \sfig<hetero-a-noise,1> the heterogeneity has a positive effect on the coherence, lowering the value of $R_T$ for low noise values
and effectively increasing the noise which also increases the frequency of excitation
and in turn shifts the optimal noise towards lower noises, though the effect vanishes for larger noises where the coherence becomes independent on the heterogeneity.

This effect is only visible close to the bifurcation since going farther away from it -- towards larger values of $a$ -- reverses it as seen in \sfig<hetero-a-noise,2>,
shifting the optimal noise towards higher noises instead, additionally the best reachable coherence also gets worse.
Nevertheless the dependence of the coherence from the heterogeneity is still only present for low noises.

\FigureRow{hetero-a-noise}{0.5}{image/plotDR_a105}{$a = 1.05$}{0.5}{image/plotDR_a13}{$a = 1.3$}{
	The Normalized standard deviation of the inter-spike-interval $R_T$ plotted over different diffusion coefficients $D$
	for small and large bifurcation parameter $a$ and for different heterogeneity $\delta a$,
	For constant coupling strength $\sigma = 0.1$ and timescale parameter $\epsilon = 0.01$.
	Simulation for $N = 100$ oscillators and constant simulation time $t = 1000$.
}

Looking at the changes over different coupling strengths $\sigma$ yields similar results. As seen in \sfig<hetero-a-coupling,1> with increasing heterogeneity $\delta a$ the
plateau of optimal coherence gets smaller as the center is shifted towards lower coupling strength but maintaining the position of its left border
and raising the plateau at low coupling strengths. Farther away from the bifurcation the effect is again reversed -- as seen in \sfig<hetero-a-coupling,2> --
as the minimum gets shifted towards higher coupling strength and the general value of coherence lessens.

\FigureRow{hetero-a-coupling}{0.5}{image/plotsR_a105}{$a = 1.05$}{0.5}{image/plotsR_a13}{$a = 1.3$}{
	The Normalized standard deviation of the inter-spike-interval $R_T$ plotted over different coupling strengths $\sigma$
	for small and large bifurcation parameter $a$ and for different heterogeneity $\delta a$,
	for optimal diffusion constant $D_{opt}$ and constant timescale parameter $\epsilon = 0.01$.
	Simulation for $N = 100$ oscillators and constant simulation time $t = 1000$.
}

In \fig<compare-space-time> an overview of these effects is given.
The top row shows that with heterogeneity (\sfig<compare-space-time,2>) there are more excitations but they start at multiple locations instead
at only one as seen in \sfig<compare-space-time,1>.
For higher bifurcation parameter -- see \sfig<compare-space-time,3> -- the amount of excitations are lower and they don't form waves anymore,
but the heterogeneity still increases the amount as shown in \sfig<compare-space-time,4>.

\FigureGrid{compare-space-time}
	{0.45}{image/plotXTD000012a105da0}{$a = 1.05, D = 0.00012, \delta a = 0$}
	{0.45}{image/plotXTD000012a105da005}{$a = 1.05, D = 0.00012, \delta a = 0.05$}
	{0.45}{image/plotXTD0004a13da0}{$a = 1.3, D = 0.004, \delta a = 0$}
	{0.45}{image/plotXTD0004a13da005}{$a = 1.3, D = 0.004, \delta a = 0.05$}{
	Space-Time plots for the homogeneous system (left) and a heterogeneous system (right) and
	low Bifurcation parameter (top) and high Bifurcation parameter (bottom) for the same parameters of
	coupling strength $\sigma = 0.1$ and timescale parameter $\epsilon = 0.01$.
}